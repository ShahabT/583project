\begin{abstract}
Applications and users are mostly oblivious to what tables are frequently
utilized and queried in the Database Management System (DMBS). 
% In the era of big-data where thousands of big-tables~\cite{googleosdi06}
As the era of big-data where thousands of bigtables are queried, utilized, and
fetched from high-latency storage, precaching tables with high-likelihood is
crucial to improve application performance. 
However, without application or user specific domain knowledge in
frequently/likely used tables, desigining a generic table precaching system is
challenging. 
In this report, we show that with static analysis of queries at compile-time,
underlying DMBS is able to identify which table to prefetch and what queries to
precache.
We design, implement, and evaluate the Speculative Database Table Precaching
(SDTP) framework that takes a generic approach in identifying and precaching
frequently/likely-used tables to mitigate traditional high-latency bottleneck
coming from non-volatile storage.
Toward the end, we claim that our approach improves the performance as close to
that of in-memory database systems, but without extra support from hardware.
\end{abstract}
